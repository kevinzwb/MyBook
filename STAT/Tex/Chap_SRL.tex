\part{增强学习}
\chapter{增强学习简介} \label{chap:introduction}


这部分是对Masashi Sugiyama的《Statistical Reinforcement Learning: Modern Machine Learning Approaches》的笔记。


增强学习试图解决未知环境下的决策问题。增强学习可以概括为:两个要素( \textbf{环境}, \textbf{玩家}),三种信号( \textbf{状态}, \textbf{行动}, \textbf{反馈})。在一个未知的环境下,玩家基于\textbf{策略}选择行动,之后,环境更新状态,并给予玩家反馈。基于环境和玩家的互动,在没有任何指令的情况下,玩家也可以完成指定任务。通常情况下,我们通过马尔可夫决策过程来建模增强学习任务。在任一离散时间$t$,玩家观察到环境状态信号$s_t \in S$,给出其行动信号$a_t \in A$,相应的,环境更新状态信号$s_{t+1} \in S$,给出即时反馈信号$r$.
$$r_t = r(s_t,a_t,s_{t+1})$$
状态信号属于状态空间,行动信号属于行动空间。$r(s_t,a_t,s_{t+1})$ 是即时反馈方程。

环境的初始状态$s_1$符合概率分布。如果状态空间是离散的,则初始概率分布可以通过概率质量函数计算$P(s)$
$$0 \leq P(s) \leq 1 \quad  \forall s \in S $$
$$\sum_{s \in S } P(s) = 1$$
如果状态空间是连续的,则初始概率分布可以通过概率密度函数计算
$$ p(s) \geq 0 \quad  \forall s \in S $$
$$\int_{s \in S} p(s) ds = 1 $$
事实上,概率质量函数可以通过概率密度函数计算(基于Dirac Delta Function),因此,之后我们只关注连续的状态空间。
$$p(s) = \sum_{s^\prime \in S} \delta(s^\prime - s)P(s^\prime)$$
环境的动态变化可以用条件概率密度表达成状态的转化概率分布(transition probability distribution)
$$p(s^\prime|s,a) \geq 0 \quad  \forall s,s^\prime \in S \quad  \forall a \in A $$
$$\int_{s^\prime \in S} p(s^\prime|s,a) ds^\prime = 1  \quad  \forall s \in S \quad  \forall a \in A $$
玩家通过\textbf{策略}$\pi$来决定行动。当采取确定策略行动时,我们可以把策略看成状态的函数。
$$\pi(s) \in A \quad  \forall s \in S $$
行动空间可以是离散的也可以是连续的。
当处理比较复杂的增强学习问题时,随机策略是更好的选择。此时,在某一状态下,玩家的行动选择存在概率分布。随机策略可以表达成行动和状态的条件概率。
$$\pi(a|s) \geq 0 \quad  \forall s \in S \quad  \forall a \in A $$
$$\int_{a \in A} \pi(a|s) da  = 1  \quad  \forall s \in S $$
随机策略可以帮助玩家更好的探索环境的所有状态。玩家和环境的一系列互动([状态,行动])被称为\textbf{经历}。

互动的次数可以是有限的,也可以是无限的。在此,我们只关注有限互动的增强学习任务。一次经历$h$可以被表达成
$$ h = [s_1,a_1,\dots,s_T,a_T,s_{T+1} $$

一次经历的\textbf{回报}$R$是衰减累计反馈。
$$R(h) = \sum_{t=1}^T\gamma^{t-1}r(s_t,a_t,s_{t+1})$$
$\gamma$ 是衰减因子。

增强学习的目的是发展出最佳策略$\pi_*$



\section{系统要求}

\texttt{ucasthesis}宏包可以在目前大多数的\TeX{}编译系统中使用,例如C\TeX{}、MiK\TeX{}、\TeX{}Live。推荐的\TeX{}编译系统 + 文本编辑器为
\begin{itemize}
    \item Linux: \TeX{}Live + vim or Texmaker
    \item MacOS: \TeX{}Live or Mac\TeX{} + Macvim or Texmaker
    \item Windows: \TeX{}Live or Mik\TeX{}  + Texmaker
\end{itemize}
\TeX{}编译系统 (如MiK\TeX{}、\TeX{}Live) 用于提供编译环境,文本编辑器 (如Texmaker、vim) 用于编辑\TeX{}源文件。

\section{问题反馈}

\begin{center}
莫晃锐 (mohuangrui) \quad mohuangrui@gmail.com

模版下载地址: \url{https://github.com/mohuangrui/ucasthesis}
\end{center}

欢迎大家反馈模板不足之处,一起不断改进模板。希望大家向同事积极推广\LaTeX{},一起更高效地做科研。
