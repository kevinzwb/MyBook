\part{增强学习}
\chapter{理论简介} 
这部分是对Masashi Sugiyama的《Statistical Reinforcement Learning: Modern Machine Learning Approaches》的笔记。
增强学习试图解决未知环境下的决策问题。增强学习可以概括为:两个要素( \textbf{环境}, \textbf{玩家}),三种信号( \textbf{状态}, \textbf{行动}, \textbf{反馈})。在一个未知的环境下,玩家基于\textbf{策略}选择行动,之后,环境更新状态,并给予玩家反馈。基于环境和玩家的互动,在没有任何指令的情况下,玩家也可以完成指定任务。通常情况下,我们通过马尔可夫决策过程来建模增强学习任务。在任一离散时间$t$,玩家观察到环境状态信号$s_t \in S$,给出其行动信号$a_t \in A$,相应的,环境更新状态信号$s_{t+1} \in S$,给出即时反馈信号$r$.
$$r_t = r(s_t,a_t,s_{t+1})$$
状态信号属于状态空间,行动信号属于行动空间。$r(s_t,a_t,s_{t+1})$ 是即时反馈方程。

环境的初始状态$s_1$符合概率分布。如果状态空间是离散的,则初始概率分布可以通过概率质量函数计算$P(s)$
$$0 \leq P(s) \leq 1 \quad  \forall s \in S $$
$$\sum_{s \in S } P(s) = 1$$
如果状态空间是连续的,则初始概率分布可以通过概率密度函数计算
$$ p(s) \geq 0 \quad  \forall s \in S $$
$$\int_{s \in S} p(s) ds = 1 $$
事实上,概率质量函数可以通过概率密度函数计算(基于Dirac Delta Function),因此,之后我们只关注连续的状态空间。
$$p(s) = \sum_{s^\prime \in S} \delta(s^\prime - s)P(s^\prime)$$

环境的动态变化可以用条件概率密度表达成状态的转移概率分布(transition probability distribution)
$$p(s^\prime|s,a) \geq 0 \quad  \forall s,s^\prime \in S \quad  \forall a \in A $$
$$\int_{s^\prime \in S} p(s^\prime|s,a) ds^\prime = 1  \quad  \forall s \in S \quad  \forall a \in A $$
玩家通过\textbf{策略}$\pi$来决定行动。当采取确定策略行动时,我们可以把策略看成状态的函数。
$$\pi(s) \in A \quad  \forall s \in S $$
行动空间可以是离散的也可以是连续的。
当处理比较复杂的增强学习问题时,随机策略是更好的选择。此时,在某一状态下,玩家的行动选择存在概率分布。随机策略可以表达成行动和状态的条件概率。
$$\pi(a|s) \geq 0 \quad  \forall s \in S \quad  \forall a \in A $$
$$\int_{a \in A} \pi(a|s) da  = 1  \quad  \forall s \in S $$
随机策略可以帮助玩家更好的探索环境的所有状态。玩家和环境的一系列互动([状态,行动])被称为\textbf{经历}。

互动的次数可以是有限的,也可以是无限的。在此,我们只关注有限互动的增强学习任务。一次经历$h$可以被表达成
$$ h = [s_1,a_1,\dots,s_T,a_T,s_{T+1} $$
一次经历$h$的\textbf{回报}$R$是衰减累计反馈。$\gamma$ 是衰减因子。
$$R(h) = \sum_{t=1}^T\gamma^{t-1}r(s_t,a_t,s_{t+1})$$


增强学习的目的是促使玩家形成最佳策略$\pi_*$,来获得最大的经历回报。
$$\pi_* = \arg \max_{\pi} \mathbb{E}_{p^{\pi}(h)} [R(h)]$$
$\mathbb{E}_{p^{\pi}(h)}$是一次经历的期望,${p^{\pi}(h)}$是策略$\pi$下的经历$h$的概率密度。
$$p^\pi(h)=p(s_1)\prod_{t=1}^T p(s_{t+1}|s_t,a_t)\pi(a_t|s_t)$$

增强学习可以分为两类,一类基于限定模型,一类基于开放模型。这里\textbf{模型}指代上面提及到的转移概率$p(s^\prime|s,a)$。限定模型的增强学习,转移概率是先验的,转移模型是显性地用于学习行动策略。开放的增强学习,玩家不能直接使用转移概率来寻找策略。一般来说,如果存在很强关于转移概率的先验知识,限定模型的方法更合适。如果不存在很好的先验知识,开放模型的增强学习更适合。

对于限定模型的增强学习,环境的转移模型是先验的,玩家不需要花费额外的精力来理解环境变化。这种类型非常适合于数据充足的任务。

\textbf{策略迭代}是开放模型下增强学习最流行的求解算法。策略迭代使用\textbf{价值函数}。价值函数$Q^\pi(s,a)$是关于状态$s$和行动$a$的函数,是某状态,行动,和策略的预期回报。
$$Q^\pi(s,a) = \mathbb{E}_{p^{\pi}(h)} [R(h)|s_1 =s,a_1=a] $$
最优价值函数是拥有最大价值的状态和行动组合$Q_*(s,a) = \max_{\pi} Q^\pi(s,a) $。最优策略$\pi_*$
$$\pi_*(a|s) = \delta(a - \arg \max_{a^\prime} Q_*(s,a^\prime) )$$
最优价值函数是未知,我们通过策略迭代来估计策略的价值$Q^\pi$,然后基于策略价值$Q^\pi$更新策略$\pi$,通过不断的迭代,是价值函数收敛到最优$Q_*(s,a)$。一般来说,策略迭代的效果取决于价值评估的效力。但是,提高价值评估的效力并不意味着策略质量的提升,而且微小的价值评估变化可能会带来不同的策略选择。此外,当行动空间是连续的情况,策略迭代算法计算量巨大,非常难以训练。
为了克服策略迭代的缺点,\textbf{策略搜寻}不采用价值评估,直接学习最大化预期回报的策略函数。在策略搜寻中,最关键的问题是如何从海量的策略空间中筛选出好的策略。
$$\pi_* = \arg \max_{\pi} \mathbb{E}_{p^{\pi}(h)} [R(h)]$$


\chapter{开放模型下的策略迭代} 
\section{价值函数的近似}
\section{近似价值函数的设计}
\section{采样复用}
\section{主动学习}
\section{稳健迭代}


\chapter{开放模型下的策略搜寻} 
\section{梯度上升算法}
\section{期望最大化算法}
\section{先验策略}

\chapter{限定模型的增强学习} 
\section{转移模型估计}
\section{维度压缩}

